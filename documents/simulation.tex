\documentclass[]{article}
\usepackage[utf8]{inputenc}
\usepackage{multicol}
\usepackage{amsmath}
\usepackage{amsthm}
\usepackage{amssymb}
\usepackage{bm}
\usepackage{physics}
\usepackage{graphicx}
\usepackage{float}
%\usepackage[round]{natbib}
%\bibliographystyle{unsrtnat}

% this just sets up the page size and margins
\setlength{\topmargin}{0pt}
\setlength{\oddsidemargin}{0pt}
\setlength{\evensidemargin}{0pt}
\setlength{\textwidth}{6.5in}
\setlength{\textheight}{8.5in}
\setlength{\headheight}{0pt}

% create some short cut commands
\newcommand{\bx}{\boldsymbol{x}}
\newcommand{\by}{\boldsymbol{y}}
\newcommand{\bu}{\boldsymbol{u}}


% this is a bunch of stuff that allows python style code snippits
% Default fixed font does not support bold face
\DeclareFixedFont{\ttb}{T1}{txtt}{bx}{n}{10} % for bold
\DeclareFixedFont{\ttm}{T1}{txtt}{m}{n}{10}  % for normal
% Custom colors
\usepackage{color}
\definecolor{deepblue}{rgb}{0,0,0.5}
\definecolor{deepred}{rgb}{0.6,0,0}
\definecolor{deepgreen}{rgb}{0,0.5,0}

\usepackage{listings}

% Python style for highlighting
\newcommand\pythonstyle{\lstset{
		language=Python,
		basicstyle=\ttm,
		morekeywords={self},              % Add keywords here
		keywordstyle=\ttb\color{deepblue},
		emph={MyClass,__init__},          % Custom highlighting
		emphstyle=\ttb\color{deepred},    % Custom highlighting style
		stringstyle=\color{deepgreen},
		frame=tb,                         % Any extra options here
		showstringspaces=false,
		tabsize=4
}}

% Python environment
\lstnewenvironment{python}[1][]
{
	\pythonstyle
	\lstset{#1}
}
{}

% Python for inline
\newcommand\pythoninline[1]{{\pythonstyle\lstinline!#1!}}

\title{Thrust Vector Drone Simulation}
\author{Izzy Mones and Heidi Dixon}
%\date{}

\begin{document}
	\maketitle
	
	\section*{Introduction}	
	\begin{enumerate}
		\item {\em Thrust Vector Drone}: What is a thrust vector drone and why they are useful control theory projects for rocket control.
		\item {\em Description:} of the drone we are simulating.
	\end{enumerate}
	
	\section*{Equations of Motion}
	\begin{enumerate}
	\item {\em Equations}:
	\item {\em Quaternions:} Why we chose to use them.
	\item {\em Thrust Model:}
	\end{enumerate}
	\section*{Algorithms}	
	
	\begin{enumerate}
	\item {\em NMPC}: What is NMPC and why did we choose it?
	\item {\em NMPC Problem Formulation:} Describe the general set of constraints. 
	\item {\em NLP encodings} Non Linear Problem Encodings
	\begin{enumerate}
		\item {\em do-mpc } What does do-mpc do
		\item {\em Single-Shooter with Runge-Kutta}: 
		\item {\em Chebyshev-Gauss-Lobatto with spectral analysis}
	\end{enumerate}
	
	\item {\em Solver:} ipopt solver, mumps, ma27 and ma57
	\item {\em Comparing Quality of Solution}  
	\item {\em Time Comparisons} 
	\end{enumerate}
	
	\subsection*{NLP Solver}
	To solve our non-linear programming (NLP) problems we used the \texttt{ipopt} solver that comes installed with CasADi.  This solver requires an additional subroutine to  solve sparse matrix systems. We experimented with the \texttt{mumps}  solver that comes with CasADi, and also tried using the  \texttt{ma27} and  \texttt{ma57} solvers  from HSL (Harwell Subroutine Library) \cite{hsl}. The HSL \texttt{ma27} solver produced the most efficient solutions overall. All solvers and experiments were run with the same solver settings.
	\vspace{\baselineskip}
	\begin{python}
        ipopt_settings = {
			'ipopt.max_iter': 100,                  
			'ipopt.tol': 1e-3,                      	
			'ipopt.acceptable_tol': 3e-2,
			'ipopt.linear_solver': 'ma27',
		}
	\end{python}
	
	\section*{Conclusions}
	
	
	
%	\nocite{*}
%	
\bibliographystyle{annotate}
\bibliography{references.bib}
\end{document}

