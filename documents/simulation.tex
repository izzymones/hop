\documentclass[]{article}
\usepackage[utf8]{inputenc}
\usepackage{multicol}
\usepackage{amsmath}
\usepackage{amsthm}
\usepackage{amssymb}
\usepackage{bm}
\usepackage{physics}
\usepackage{graphicx}
\usepackage{float}
%\usepackage[round]{natbib}
%\bibliographystyle{unsrtnat}

% this just sets up the page size and margins
\setlength{\topmargin}{0pt}
\setlength{\oddsidemargin}{0pt}
\setlength{\evensidemargin}{0pt}
\setlength{\textwidth}{6.5in}
\setlength{\textheight}{8.5in}
\setlength{\headheight}{0pt}

% create some short cut commands
\newcommand{\bx}{\boldsymbol{x}}
\newcommand{\by}{\boldsymbol{y}}
\newcommand{\bu}{\boldsymbol{u}}


% this is a bunch of stuff that allows python style code snippits
% Default fixed font does not support bold face
\DeclareFixedFont{\ttb}{T1}{txtt}{bx}{n}{10} % for bold
\DeclareFixedFont{\ttm}{T1}{txtt}{m}{n}{10}  % for normal
% Custom colors
\usepackage{color}
\definecolor{deepblue}{rgb}{0,0,0.5}
\definecolor{deepred}{rgb}{0.6,0,0}
\definecolor{deepgreen}{rgb}{0,0.5,0}

\usepackage{listings}

% Python style for highlighting
\newcommand\pythonstyle{\lstset{
		language=Python,
		basicstyle=\ttm,
		morekeywords={self},              % Add keywords here
		keywordstyle=\ttb\color{deepblue},
		emph={MyClass,__init__},          % Custom highlighting
		emphstyle=\ttb\color{deepred},    % Custom highlighting style
		stringstyle=\color{deepgreen},
		frame=tb,                         % Any extra options here
		showstringspaces=false,
		tabsize=4
}}

% Python environment
\lstnewenvironment{python}[1][]
{
	\pythonstyle
	\lstset{#1}
}
{}

% Python for inline
\newcommand\pythoninline[1]{{\pythonstyle\lstinline!#1!}}

\title{Thrust Vector Drone Simulation}
\author{Izzy Mones and Heidi Dixon}
%\date{}

\begin{document}
	\maketitle
	
	\section*{Introduction}	
	\begin{enumerate}
		\item {\em Thrust Vector Drone}: What is a thrust vector drone and why they are useful control theory projects for rocket control.
		\item {\em Description:} of the drone we are simulating.
	\end{enumerate}
	
	\section*{Equations of Motion}
	\begin{enumerate}
	\item {\em Equations}:
        \[
        \dot{\vec{x}} =
        \begin{bmatrix}
        x \\ y \\ z \\ v_x \\ v_y \\ v_z \\ q_x \\ q_y \\ q_z \\ q_w \\ \omega_x \\ \omega_y \\ \omega_z
        \end{bmatrix}
        \]
        \[
        \dot{\vec{u}} =
        \begin{bmatrix}
        \theta_1 \\ \theta_2 \\ T_{avg} \\ T_{diff}
        \end{bmatrix}
        \]
	\[
		\dot{\vec{p}} = \vec{v}
	\]
	\[
		\dot{\vec{v}} = \frac{1}{m}R(\vec{q})\vec{F_b}+\vec{g}
	\]
	\[
		\dot{\vec{q}} = \frac{1}{2}Q(\vec{\omega})\vec{q}
	\]
	\[
		\dot{\vec{\omega}} = I^{-1}\!\left(\vec{M_b} - \vec{\omega} \times (I\,\vec{\omega})\right)
	\]
	\[
		\dot{x} = v_x
	\]
	\[
		\dot{y} = v_y
	\]
	\[
		\dot{z} = v_z
	\]
	 \[
         \vec{F_b} = T
        \begin{bmatrix}
        \sin{\theta_2}  \\
         -\sin{\theta_1}\cos{\theta_2}  \\
         \cos{\theta_1}\cos{\theta_2}
        \end{bmatrix}
        \]
        \[
	\vec{l} =
        \begin{bmatrix}
        0  \\
        0  \\
        -l
        \end{bmatrix}
        \]
        
        \[
	\vec{M_z} =
        \begin{bmatrix}
        0  \\
        0  \\
        M_z
        \end{bmatrix}
        \]
        	 \[
        \vec{M_b} = \vec{l} \times \vec{F_b} + \vec{M_z}=       
        \begin{bmatrix}
        -lT\sin{\theta_1}\cos{\theta_2}  \\
        -lT\sin{\theta_2}  \\
	M_z
        \end{bmatrix}
        \]
        
         \[
        I =
        \begin{bmatrix}
        I_{xx} & 0 &0 \\
        0 & I_{yy }& 0 \\
        0 & 0 & I_{zz}
        \end{bmatrix}
        \]

        \[
        R(\vec{q}) =
        \begin{bmatrix}
        1 - 2(q_y^2 + q_z^2) & 2(q_x q_y - q_w q_z) & 2(q_x q_z + q_w q_y) \\
        2(q_x q_y + q_w q_z) & 1 - 2(q_x^2 + q_z^2) & 2(q_y q_z - q_w q_x) \\
        2(q_x q_z - q_w q_y) & 2(q_y q_z + q_w q_x) & 1 - 2(q_x^2 + q_y^2)
        \end{bmatrix}
        \]
        
        \[
        Q(\vec{\omega}) =
        \begin{bmatrix}
        0 & \omega_z & -\omega_y & \omega_x \\
        -\omega_z & 0 & \omega_x & \omega_y \\
        \omega_y & -\omega_x & 0 & \omega_z \\
        -\omega_x & -\omega_y & -\omega_z & 0
        \end{bmatrix}
	\]
	\[
		\dot{x} = v_x
	\]
	\[
		\dot{y} = v_y
	\]
	\[
		\dot{z} = v_z
	\]
	\[
		\dot{v_x} = \frac{\sin{\theta_2} *T(1 - 2(q_y^2 + q_z^2))-\sin{\theta_1}\cos{\theta_2}*2T(q_x q_y - q_w q_z)+\cos{\theta_1}\cos{\theta_2}* 2T(q_x q_z + q_w q_y)}{m}
	\]
	\[
		\dot{v_y} =\frac{\sin{\theta_2} *2T(q_x q_y + q_w q_z)-\sin{\theta_1}\cos{\theta_2}*T(1 - 2(q_x^2 + q_z^2))+\cos{\theta_1}\cos{\theta_2}* 2T(q_y q_z - q_w q_x)}{m}+g
	\]
	\[
		\dot{v_z} = \frac{\sin{\theta_2} *2T(q_x q_z - q_w q_y-\sin{\theta_1}\cos{\theta_2}* 2T(q_y q_z + q_w q_x)+\cos{\theta_1}\cos{\theta_2}* T(1 - 2(q_x^2 + q_y^2))}{m}
	\]
	\[
		\dot{q_x} = \frac{\omega_z q_y-\omega_y q_z+ \omega_x q_w}{2}
	\]
	\[
		\dot{q_y} = \frac{-\omega_z q_x+\omega_x q_z+ \omega_y q_w}{2}
	\]
	\[
		\dot{q_z} =  \frac{\omega_y q_x-\omega_x q_y+ \omega_z q_w}{2}
	\]
	\[
		\dot{q_w} =  \frac{-\omega_x q_x - \omega_y q_y - \omega_z q_z}{2}
	\]
	\[
		\dot{\omega_x} = \frac{-lT\sin{\theta_1}\cos{\theta_2}-\omega_y\omega_zI_{zz} + \omega_y\omega_zI_{yy}}{I_{xx}}
	\]
	\[
		\dot{\omega_y} = \frac{-lT\sin{\theta_2}-\omega_x\omega_zI_{xx} + \omega_x\omega_zI_{zz}}{I_{yy}}
	\]
	\[
		\dot{\omega_z} =  \frac{M_z -\omega_x\omega_yI_{yy} + \omega_x\omega_yI_{xx}}{I_{zz}}
	\]
	\item {\em Quaternions:} Why we chose to use them.
	\item {\em Thrust Model:}
	\[
		T = aT_{avg}^2+bT_{avg}+c
	\]
	\[
		M_z = d I_{zz} T_{diff}
	\]
	\[
		T_1 = T_{avg} + T_{diff}/2
	\]
	\[
		T_2 = T_{avg} - T_{diff}/2
	\]
		\end{enumerate}
	
	\section*{Nonlinear Model Predictive Control (NMPC)}
	
	\begin{enumerate}
	\item {\em NMPC}: What is NMPC and why did we choose it?
	\item {\em NMPC Problem Formulation:} Describe the general set of constraints. 
        \[
        \min_{u(t)} \; \int_{t_0}^{t_f} \ell(x,u,t)\,dt \;+\; \phi(x_f)
        \]
        \[
	\ell(x,u,t) = (x(t)-x_{ref})^T Q (x(t)-x_{ref}) + (u(t)-u_{ref})^T R (u(t)-u_{ref}) 
        \]
        \[
	\phi(x_f) = (x(t_f)-x_{ref})^T Q_f (x(t_f)-x_{ref})
        \]
        \[
	-\theta_{max} \leq \theta_1 \leq \theta_{max}
        \]
        \[
	-\theta_{max} \leq \theta_2 \leq \theta_{max}
        \]
        \[
        	-\dot{\theta}_{max} \leq \dot{\theta}_1 \leq \dot{\theta}_{max}
        \]
        \[
        	-\dot{\theta}_{max} \leq \dot{\theta}_2 \leq \dot{\theta}_{max}
        \]
        \[
        	T_{min}  \leq T_{avg} + T_{diff}/2 \leq T_{max}
        \]
        \[
        T_{min} \leq  T_{avg} - T_{diff}/2 \leq T_{max}
        \]
        \[
        T_{diff_{min}} \leq T_{diff} \leq T_{diff_{max}}
        \]
        \[
        0 \leq z
        \]
        
	\item {\em NLP encodings} Non Linear Problem Encodings
	\begin{enumerate}
		\item {\em do-mpc } What does do-mpc do
		\item {\em Single-Shooter with Runge-Kutta}: 
	\[
        \begin{aligned}
        k_1 &= f(x_k, u_k, t_k), \\[6pt]
        k_2 &= f\!\left(x_k + \tfrac{\Delta t}{2} k_1,\, u_k,\, t_k + \tfrac{\Delta t}{2}\right), \\[6pt]
        k_3 &= f\!\left(x_k + \tfrac{\Delta t}{2} k_2,\, u_k,\, t_k + \tfrac{\Delta t}{2}\right), \\[6pt]
        k_4 &= f\!\left(x_k + \Delta t\, k_3,\, u_k,\, t_k + \Delta t\right), \\[10pt]
        x_{k+1} &= x_k + \tfrac{\Delta t}{6}\,\Big(k_1 + 2k_2 + 2k_3 + k_4\Big).
        \end{aligned}
        \]
		\item {\em Chebyshev-Gauss-Lobatto with spectral analysis}
	\[
        \tau_j = \cos{(\frac{j\pi}{N})},\qquad j = 0,1,...,N
        \]
	\[
       D_{N_{00}}=\frac{2N^2+1}{6}
        \]
	\[
       D_{N_{jj}}=\frac{-x_j}{2(1-x_j^2)}
        \]
	\[
       D_{N_{ij}}=\frac{c_i}{c_j}\frac{(-1)^{i+j}}{x_i-x_j}
        \]
	\[
       D_{N_{NN}}=-\frac{2N^2+1}{6}
        \]
	\[
        c_{i/j}=
        \begin{cases}
        2 & i/j = 0 \;\text{or}\; N, \\
        1 & \text{otherwise}.
        \end{cases}
        \]
        \[
        D_{6} =
        \begin{bmatrix}
        \frac{73}{6} & -(8+4\sqrt{3})  & 4 & -2 & \frac{4}{3} & 4\sqrt{3} - 8 & \frac{1}{2}\\
        2+\sqrt{3} & -\sqrt{3} & -(1+\sqrt{3}) & \frac{2}{\sqrt{3}} & 1-\sqrt{3} & \frac{1}{\sqrt{3}} & \frac{1-\sqrt{3}}{1+\sqrt{3}}\\
        -1 & 1+\sqrt{3} & -\frac{1}{3} & -2 & 1 & 1-\sqrt{3} & \frac{1}{3}\\
        \frac{1}{2} & -\frac{2}{\sqrt{3}} & 2 & 0 & -2 & \frac{2}{\sqrt{3}} & -\frac{1}{2}\\
        -\frac{1}{3} & \sqrt{3}-1 & -1 & 2 & \frac{1}{3} & -(1+\sqrt{3}) & 1\\
        \frac{1-\sqrt{3}}{1+\sqrt{3}} & \sqrt{3} & \sqrt{3}-1 & -\frac{2}{\sqrt{3}} & 1+\sqrt{3} & \sqrt{3} &-( 2+\sqrt{3})\\
        -\frac{1}{2} & 8-4\sqrt{3} & -\frac{4}{3} & 2 & -4 & 8+4\sqrt{3} &  -\frac{73}{6}
        \end{bmatrix}
        \]
        \[
        \dot{x}(\tau_i) \;\approx\; \sum_{j=0}^N D_{ij}\,x(\tau_j), 
        \qquad i=0,\dots,N.
        \]
        \[
        t(\tau) = \frac{t_f-t_0}{2}\,\tau + \frac{t_f+t_0}{2}, 
        \qquad 
        \frac{dt}{d\tau} = \frac{t_f-t_0}{2}.
        \]
        
        \[
        \sum_{j=0}^N D_{ij}\,x_j 
        = \frac{t_f-t_0}{2}\, f(x_i,u_i), 
        \qquad i=1,\dots,N-1.
        \]
        

        \[
        w_{\mathrm{j_{even}}} =
        \begin{cases}
        \displaystyle
        \frac{2}{N}\!\left(
        1 - \sum_{n=1}^{\frac{N}{2}-1} \frac{2}{4n^{2}-1}\,
        \cos\!\Big(\tfrac{2 n j \pi}{N}\Big)
        -\frac{(-1)^{j}}{N^{2}-1}
        \right), & 1 \le j \le N-1, \\[2.0ex]
        \displaystyle \frac{1}{N^{2}-1}, & j=0\ \text{or}\ j=N \quad(\text{since \(N\) even}).
        \end{cases}
        \]
	\end{enumerate}
	\[
        w_{\mathrm{j_{odd}}} =
        \begin{cases}
        \displaystyle
        \frac{2}{N}\!\left(
        1 - \sum_{n=1}^{\frac{N}{2}-1} \frac{2}{4n^{2}-1}\,
        \cos\!\Big(\tfrac{2 n j \pi}{N}\Big)
        \right), & 1 \le j \le N-1, \\[2ex]
        \displaystyle \frac{1}{N^{2}}, & j=0\ \text{or}\ j=N.
        \end{cases}
        \]
        	\[
	J=\frac{t_f-t_0}{2}\int_{t_0}^{t_f} \ell(x(t),u(t),t)\,dt \;+\; \phi(x(t_f))
        \]
        \[
        J \;\approx\; \frac{t_f-t_0}{2} \sum_{j=0}^{N} w_j \,\ell(x_j,u_j,t_j)
        \;+\; \phi(x_N),
        \qquad t_j = \frac{t_f-t_0}{2}\,\tau_j + \frac{t_f+t_0}{2}.
        \]
        	
	\item {\em Solver:} ipopt solver, mumps, ma27 and ma57
	\item {\em Comparing Quality of Solution}  
	\item {\em Time Comparisons} 
	\end{enumerate}
	

	
	
	\subsection*{NLP Solver}
	To solve our non-linear programming (NLP) problems we used the \texttt{ipopt} solver that comes installed with CasADi.  This solver requires an additional subroutine to  solve sparse matrix systems. We experimented with the \texttt{mumps}  solver that comes with CasADi, and also tried using the  \texttt{ma27} and  \texttt{ma57} solvers  from HSL (Harwell Subroutine Library) \cite{hsl}. The HSL \texttt{ma27} solver produced the most efficient solutions overall. All solvers and experiments were run with the same solver settings.
	\vspace{\baselineskip}
	\begin{python}
        ipopt_settings = {
			'ipopt.max_iter': 100,                  
			'ipopt.tol': 1e-3,                      	
			'ipopt.acceptable_tol': 3e-2,
			'ipopt.linear_solver': 'ma27',
		}
	\end{python}
	
	\section*{Conclusions}
	
	
	
%	\nocite{*}
%	
\bibliographystyle{annotate}
\bibliography{references.bib}
\end{document}

